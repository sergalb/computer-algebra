\documentclass[14pt,a4paper,]{article}
\usepackage[utf8]{inputenc}

\usepackage[russian]{babel}
\usepackage{amsmath}
\usepackage{amsfonts}
\usepackage{amssymb}
\usepackage{graphicx}
\usepackage{hyperref}

\title{Метод Диксона разложения целых чисел на множители с использованием метода сопряженных градиентов решения СЛАУ }
\date{}
\author{Балахнин Сергей}





% Основные математические символы
\newcommand{\N}{\mathbb{N}}   % Natural numbers
\newcommand{\R}{\mathbb{R}}   % Ratio numbers
\newcommand{\Z}{\mathbb{Z}}   % Integer numbers
\newcommand{\F}{\mathbb{F}}   % Binary field
\def\EPS{\varepsilon}         %
\def\SO{\Rightarrow}          % =>
\def\EQ{\Leftrightarrow}      % <=>
\def\t{\texttt}               % mono font
\def\O{\mathcal{O}}           %
\def\A{\mathcal{A}}
\def\XOR{\text{ {\raisebox{-2pt}{\ensuremath{\Hat{}}}} }}
\newcommand{\sfrac}[2]{{\scriptstyle\frac{#1}{#2}}}  % Очень маленькая дробь
\newcommand{\mfrac}[2]{{\textstyle\frac{#1}{#2}}}    % Небольшая дробь

\begin{document}

\maketitle
\section{Описание метода Диксона}
Метод Диксона является обобщением метода факторизации Ферма:\\
Пусть n - нечетное число. Тогда можно найти такие s, t, что $n = s^2 - t^2 = (s+t)(s - t)$ \\
Теперь давайте искать s, t, такие что $s ^ 2 \equiv t^ 2 \pmod{n}$\\
тогда $ (s + t)(s - t) \equiv 0 \pmod{n}  \SO \gcd(s + t, n); \gcd(s - t, n)$ являются делителями n и, если $ s \not\equiv \pm t \pmod{n}$, то они являются не тривиальными делителями n. 
Опишем способ нахождения s и t:\\
Возьмем множество простых чисел  $B = \{ p_1, ... p_h\}$ и будем случайно выбирать числа b.
Назовем число b B-гладким, если $ b^2 \pmod n$ является произведением простых чисел из B.
Сделаем из B-гладких чисел и $p_i$ числа s и t:\\
Пусть $b^2 \equiv p_1^{a_1} p_2^{a_2} ... p_h^{a_h} \pmod{n}, a_1...a_h \in \N$.\\
Для каждого b определим $\EPS = (a_1 \pmod 2, a_2 \pmod 2, ... a_h \pmod 2) \in \F_2^h$ 
найдем такие $b_1, b_2, ... b_l$, что $\EPS_1 + \EPS_2 + ... +\EPS_l = 0$. При $l \geq h+1 $ соответствующее множество $\EPS_i$ точно найдется, так как это множество будет ЛЗ в поле $\F_2^h$ \hfill $(*)$\\
Рассмотрим $b_i^2 = \prod_{j = 1}^{h} p_j^{a_{i}_{j}}$\\
определим $\gamma_j = \frac{1}{2} \sum_{i=1}^{l} a_{i}_{j}$ (так как для каждого $p_j$ сумма $a_j$ по всем $b_{1..l}$ - четная, то $\gamma_j \in \N$).\\

$$(\prod_{i = 1}^{l} b_i) ^ 2 =\prod_{j=1}^{h}p_j^{\sum_{k = 1}^{l} a_{k}_{j}} =  \prod_{j = 1}^{h} p_j^{2\gamma_j} = (\prod_{j = 1}^{h} p_j^{\gamma_j})^2$$
Соответственно $s = \prod_{i = 1}^{l} b_i \pmod n $, $t = (\prod_{j = 1}^{h} p_j^{\gamma_j}) \pmod n$
но эти s и t могут быть равны по модулю n, тогда они не помогут для решения задачи факторизации.
Какова вероятность такого исхода? Пусть n является произведением степеней r различных простых чисел. Из китайской теоремы об остатках количество корней квадрата в $\Z_n$ равно $2^r$, т.е. $prob(s \equiv \pm t \pmod n = \frac{2}{2^r})$, при $r \geq 2$, то есть, если N составное число, то вероятность получить $s \equiv \pm t \pmod n$ не больше  $\frac{1}{2}$.

\section{Метод сопряженных градиентов решения СЛАУ}
Для поиска s необходимо было найти такое подмножество $\{b_i\}, i = 1 .. h + 1$, что соответствующая сумма $\EPS_i$ была нулевой, то есть решить систему линейных уравнений на векторах $\EPS$. От выбора способа решения зависит скорость работы алгоритма. \\
Рассмотрим метод сопряженных градиентов решения СЛАУ
Пусть дана система уравнений, записанная в матричном ввиде
$$ Ax = \omega $$ 
A - симметричная положительно определенная матрица. \\
Тогда решение этой СЛАУ совпадает с минимумом функционала $$ (Ax, x) - 2(\omega, x) -> min $$
Итеративный алгоритм:\\
Подготовка
\begin{enumerate}
    \item выберем начальное приближение $x^0$
    \item $r^0 = \omega - Ax^0$
    \item $z^0 = r^0$
\end{enumerate}
Описание k-ой итерации
\begin{enumerate}
    \item $a^k = \frac{(r^{k-1}, r^{k-1})}{(Az^{k-1}, z^{k-1})} $
    \item $ x^k = x^{k-1} + a_k z^{k-1}$
    \item $ r^k = r^{k-1} - a_k A z^{k-1}$ 
    \item $ \beta ^ k = \frac{(r^k, r^k)}{(r^{k-1}, r^{k-1})}$
    \item $ z^k = r^k + \beta_k z^{k-1}$
\end{enumerate}
Однако в нашей задаче матрица не симметрична, а СЛАУ имеет вид $Bx = u$, где B - матрица  h x (h + 1), причем матрица B разреженная 
получим матрицу А для данного метода следующим способом:
Выберем конечное поле F с $ |F| >> n $ и сгенерируем диагональную матрицу D h x h с элементами на диагонали, выбранными случайно из поля $F\textbackslash\{0\}$  \\
$ A = B^T D^2 B$\\
$ \omega = B^T D^2 u $\\
с высокой вероятностью решения изначального и преобразованного уравнений будут равны. В случае если это не так, нужно сгенерировать новую матрицу D. Также проблемой могут стать самоортагональные вектора, в поле F. Чтобы ее избежать обычно в качестве поля F выбирают поле $GF(2^r)$ где r=19..21.
При реализации данного метода в качестве оптимизации может использоваться хранение разреженной матрицы B как списка строк в которых есть ненулевые элементы.
\section{Архитектура программы}
Планируется разбить решение на 2 файла: класс разреженной матрицы, в которой будет реализован метод сопряженных векторов, а также основной файл с алгоритмом Диксона.\\
В основном файле будет создана фукция factorize, которая в цикле выполняет следующие действия:
\begin{enumerate}
    \item Создает список простых чисел
    \item генерирует B-гладкие числа на основе простых
    \item создает СЛАУ из векторов $\EPS$ и решает ее используя класс SparseMatrix
    \item получает числа s и t с помощью решения СЛАУ, проверяет, подходят ли они, заканчивает если да, возвращается к пункту один, если нет
\end{enumerate}
\section{Литература}
\begin{enumerate}
    \item Modern Computer Algebra, Joachim Von Zurgathen
    \item Solving Large Sparse Linear Systems Over Finite Fields B. A. LaMacchia; A.M. Odlyzko
    \item \href{https://ru.wikipedia.org/wiki/%D0%9C%D0%B5%D1%82%D0%BE%D0%B4_%D1%81%D0%BE%D0%BF%D1%80%D1%8F%D0%B6%D1%91%D0%BD%D0%BD%D1%8B%D1%85_%D0%B3%D1%80%D0%B0%D0%B4%D0%B8%D0%B5%D0%BD%D1%82%D0%BE%D0%B2_(%D0%B4%D0%BB%D1%8F_%D1%80%D0%B5%D1%88%D0%B5%D0%BD%D0%B8%D1%8F_%D0%A1%D0%9B%D0%90%D0%A3)}{Метод сопряжённых градиентов (для решения СЛАУ)}, Википедия
\end{enumerate}
\end{document}
